\documentclass[11pt,a4paper]{article}
\usepackage[utf8]{inputenc}
\usepackage{amsmath}
\usepackage{amsfonts}
\usepackage{amssymb}
\usepackage{hyperref}
\usepackage{geometry}
\usepackage{booktabs}

\geometry{margin=1in}

% --- SOVEREIGN HEADER BLOCK ---
\begin{document}

\begin{flushleft}
\textbf{The Kish Lattice Project} \\
\textbf{Copyright:} © 2026 Timothy John Kish \\
\textbf{Primary Theory DOI:} \href{https://doi.org/10.5281/zenodo.18383486}{10.5281/zenodo.18383486} \\
\textbf{Resolution DOI:} \href{https://doi.org/10.5281/zenodo.18408650}{10.5281/zenodo.18408650} \\
\textbf{Source Repository:} \href{https://github.com/TimothyKish/Holographic-Resonance-The-Geometry-of-a-Quantized-Universe}{GitHub: KishLattice/Source} \\
\rule{\linewidth}{0.5pt}
\end{flushleft}

\begin{center}
    \Large \textbf{Kish Resonance Drive: Operations Manual and Lab Guide} \\
    \vspace{0.2cm}
    \large \textit{Standard Operating Procedures for Clock-Lock and Local Agency Offset ($\sigma$)}
\end{center}

\section{Introduction}
This manual provides the technical specifications for operating a Resonance Drive within a $16/\pi$ stiff vacuum. Successful operation requires matching the craft's internal frequency to the local nodal density of the lattice, accounting for planetary and biological interference.

\section{The Local Agency Offset ($\sigma$)}
The vacuum is a universal constant, but the "Agency of Life" and planetary mass create localized refractions in nodal density. To achieve a stable slipstream, the operator must apply a \textbf{Local Agency Offset}.

\subsection{Planetary Tuning Reference}
The following table provides the required offsets derived from WMAP anisotropy data and nodal density observations:

\begin{table}[h!]
\centering
\begin{tabular}{@{}lll@{}}
\toprule
\textbf{Location} & \textbf{Agency Offset ($\sigma$)} & \textbf{Primary Correction} \\ \midrule
Deep Space (Void) & 1.00000 & 16/$\pi$ Raw Harmonic \\
Earth (Low Orbit) & 1.00042 & +4.2 MHz Bio-Agency Shift \\
Mars (Low Orbit) & 0.99815 & -1.8 MHz Low-Agency Shift \\
Moon (Surface) & 0.98210 & -17.9 MHz Static Node Shift \\ \bottomrule
\end{tabular}
\caption{Required Offsets for Resonant Clock-Lock}
\end{table}



\section{Operational SOP: Achieving Slipstream}
\begin{enumerate}
    \item \textbf{Sensor Ping:} Utilize the Muon sensor array to detect the local nodal echo. 
    \item \textbf{Delta Calculation:} Calculate the difference between the local echo and the $16/\pi$ reference to find the $\sigma$ value.
    \item \textbf{Clock-Lock:} Adjust the 2D Prime Clock to the offset frequency.
    \item \textbf{Hull Resonation:} Activate actuators. When the Muon-detected "Wobble" reaches the Null Zone, the craft is in a state of \textbf{Geometric Transparency}.
\end{enumerate}

\section{Laboratory Guide for 16/$\pi$ Verification}
Bench-testing requires a high-vacuum torsion balance and a Muon-coincidence counter.
\begin{itemize}
    \item \textbf{The Goal:} Demonstrate a reduction in inertial decay when the resonator is tuned to the local agency offset.
    \item \textbf{Observation:} A successful test will show a sharp drop in power consumption as the drive stops "fighting" the lattice and begins slipstreaming.
\end{itemize}

\end{document}